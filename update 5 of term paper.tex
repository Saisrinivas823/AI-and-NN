\documentclass{article}
\usepackage[utf8]{inputenc}

\title{Deep learning}
\author{Submitted By Lingamgunta saikumar}
\date{18th oct 2021}

\begin{document}

\maketitle

\section{}
This is a form of machine learning uses both supervised and unsupervised and subset of 
machine learning and AI. It uses the method of artificial neural network (ANN) with 
representation learning. ANN is inspired by the human brain neural network system 
whether human brain network is dynamic (Plastic) and analog at the same time the ANN 
is static and symbolic. It can learn, memorize, generalized and prompted modelling of 
biological neural system. ANNs are more effective to solve problems related to pattern 
recognition and matching, clustering and classification. The ANN consist of standard three 
layer input, output and hidden layer, the output layer can be the input layer for the next 
output the simple network of neural system ,if there many hidden layer are present that 
ANN known as Deep Neural Networks”, or briefly DNN, can be successfully expert to solve 
difficult problems. Deep learning models yield results more quickly than standard machine 
learning approaches. 
Example: EEG based pattern recognition which uses brain computer Interface (BCI) to 
control prosthetic arm, Neuroprosthesis etc.
Other artificial intelligence techniques :
Artificial Intelligence is the intelligence of machine that simulates the human intelligence 
which programmed in such way that it thinks and act like human. It includes; reasoning, 
knowledge representation, planning, learning, natural language processing, perception, 
the ability to move and manipulate objects and many more subjects. AI has four main 
components Expert systems, Heuristic problem solving, Natural Language Processing (NLP) 
and Vision. In human the intelligent agents like eyes, ears, and other organs act as 
sensors, and hands, legs, mouth, and other body parts act as per instruction known as 
effectors similarly the robotic agent substitutes cameras and infrared range finders for the 
sensors and various motors for the effectors. A software agent has encoded bit strings as 
its precepts and actions. Similarity between human and artificial intelligence is shown in 
Table 1. AI can be divided into two categories as per its function as symbolic learning (SL) 
and machine learning (ML). SL is perform the functions like image processing through 
computer vision and understands the environment through robotics. ML computes the 
large amount of data to get a solution to the problem in terms of pattern recognition. 
Statistical machine learning embedded with speech recognition and natural language 
processing. Deep learning recognizes objects by computer vision through convolution 
neural network (CNN) and memorize past by recurrent neural network (RNN).
The methods or techniques used for the AI are classifier and prediction. Classifier is an 
algorithm that implements classification; the classifiers are Perceptron, Naïve Bayes, 
Decision trees, Logistic regression, K nearest Neighbour, AANN/DL and support vector 
machine . Perceptron is the basic building block of the neural network it breakdown the 
complex network to smaller and simpler pieces. The classifier used in the myoelectric 
Update 5 
prosthetic hand is LDA classifier, Quadratic discriminant classifier and Multilayer 
perceptron neural network with linear activation functions etc. LDA (linear discriminant 
classifier) is a simple one that helps to reduce the dimension of the algorithm for 
application of neural network model. Prediction is a method to predict a pattern an 
output noise free data with a model from input data in hidden layer.
Examples: EMG CNN based prosthetic hand, EGG based Mind controlled prosthesis with 
sensory feedback, robotic arm, exoskeleton Orthosis.
Artificial intelligence in prosthetics and orthotics :
Implementation of artificial intelligence in controlling prostheses has increased drastically 
and thus enables the amputee to operate the prosthesis more desirably. Adaptive 
controlling would enable a system to perform closer to the desired output by adjusting 
the input with the help of a feedback system. Recently, a mind-controlled limb (type of 
myoelectric controlling) was introduced as the latest advancement in the artificial 
intelligence-aided control system. A joint project between the Pentagon and Johns 
Hopkins Applied Physics Laboratory (APL) has come up with a modular prosthetic limb 
which would be fully controlled by sensors implanted in the brain, and would even restore 
the sense of touch by sending electrical impulses from the limb back to the sensory cortex 
[25]. Chang et al. (2009) proposed a multilayer artificial neural network (ANN)-based 
model to discover the essential correlation between the intrinsic impaired neuromuscular 
activities of people with spina bifida (SB) and their extrinsic gait behaviors [26]. The 
application of AI in prosthetics and orthotics is divided into various subparts according to 
the involvement of the region that get affected i.e. Lower extremity prosthesis and 
Orthosis, Upper extremity Orthosis and prosthesis, and rehabilitation aids like motorized 
mobility devices.
\end{document}
