\documentclass{article}
\usepackage[utf8]{inputenc}

\title{hidden morkov model}
\author{Submitted By Lingamgunta saikumar}
\date{15,sep,2021}

\begin{document}

\maketitle

\section{}
Hidden Markov Model(HMM)is a statistical Markov model in which the system 
beingmodeled is assumed to be a Markov process. 
Learning :
The parameter learning task in HMMs is to find, given an output sequence or a set of such 
sequences, the best set of state transition and emission probabilities. The task is usually to 
derive the maximum likelihood estimate of the parameters of the HMM given the set of 
output sequences. No tractable algorithm is known for solving this problem exactly, but a 
local maximum likelihood can be derived efficiently using the Baum–Welch algorithm or the 
Baldi–Chauvin algorithm. The Baum–Welch algorithm is a special case of the expectationmaximization algorithm. If the HMMs are used for time series prediction, more 
sophisticated Bayesian inference methods, like Markov chain Monte Carlo (MCMC) sampling 
are proven to be favorable over finding a single maximum likelihood model both in terms of 
accuracy and stability. Since MCMC imposes significant computational burden, in cases 
where computational scalability is also of interest, one may alternatively resort to 
variational approximations to Bayesian inference, Indeed, approximate variational inference 
offers computational efficiency comparable to expectation-maximization, while yielding an 
accuracy profile only slightly inferior to exact MCMC-type Bayesian inference.
History :
Hidden Markov Models were described in a series of statistical papers by Leonard E. Baum 
and other authors in the second half of the 1960s. One of the first applications of HMMs 
was speech recognition, starting in the mid-1970s.
In the second half of the 1980s, HMMs began to be applied to the analysis of biological 
sequences, in particular DNA. Since then, they have become ubiquitous in the field of 
bioinformatics.
Applications :
HMMs can be applied in many fields where the goal is to recover a data sequence that is not 
immediately observable (but other data that depend on the sequence are).
Applications include: computational finance , single molecule –kinetic analysis,
cryptanalysis,
Speech recognition , speech synthesis , parts of speech tagging , machine translation, 
chromatin state discovery , transportation forecast.
Inference :
Several inference problems are associated with hidden Markov models, as outlined below .
Probability of latent variables :
Filtering :
The task is to compute, given the model's parameters and a sequence of observations, the 
distribution over hidden states of the last latent variable at the end of the sequence . This 
task is normally used when the sequence of latent variables is thought of as the underlying 
states that a process moves through at a sequence of points of time, with corresponding 
observations at each point in time. Then, it is natural to ask about the state of the process at 
the end. 
This problem can be handled efficiently using the forward algorithm.
Most likely explanation :
The task, unlike the previous two, asks about the joint probability of the entire sequence of 
hidden states that generated a particular sequence of observations (see illustration on the 
right). This task is generally applicable when HMM's are applied to different sorts of 
problems from those for which the tasks of filtering and smoothing are applicable. An 
example is part-of-speech tagging, where the hidden states represent the underlying parts 
of speech corresponding to an observed sequence of words. In this case, what is of interest 
is the entire sequence of parts of speech, rather than simply the part of speech for a single 
word, as filtering or smoothing would compute. This task requires finding a maximum over 
all possible state sequences, and can be solved efficiently by the Viterbi algorithm.
Statistical significance :
For some of the above problems, it may also be interesting to ask about statistical 
significance. What is the probability that a sequence drawn from some null distribution will 
have an HMM probability (in the case of the forward algorithm) or a maximum state 
sequence probability (in the case of the Viterbi algorithm) at least as large as that of a
particular output sequence? When an HMM is used to evaluate the relevance of a 
hypothesis for a particular output sequence, the statistical significance indicates the false 
positive rate associated with failing to reject the hypothesis for the output sequence.

\end{document}
