\documentclass{article}
\usepackage[utf8]{inputenc}

\title{Gödel's incompleteness theorems}
\author{Submitted By Lingamgunta saikumar}
\date{21,July 2021}

\begin{document}

\maketitle

\section{Introduction}
Gödel's incompleteness theorems : 
Theseare two theorems ofmathematical logic that are concerned with the limits of 
provability in formal axiomatic theories.And the results of these are important in 
mathematical logic and the philosophy of mathematics . these theorems are widely , but not 
universally, interpreted as showing that Hilbert’s program to find a complete and consistent 
set of axioms for all mathematics is impossible .
THE FIRST INCOMPLETENESS THEOREM states that no consistent systemof axioms whose 
theorems can be listed by aneffective procedure is capable of proving all truths about the 
arithmetic of natural numbers. For any such consistent formal system, there will always be 
statements about natural numbers that are true, but that are unprovable within the system. 
THE SECOND INCOMPLETENESS THEOREM an extension of the first, shows that the system 
cannot demonstrate its own consistency.
This theorem is stronger than first incompletenesstheorem because the statement 
constructed in the first incompleteness theorem does not directly express the consistency 
of the system .
Employing a diagonal argument, Gödel's incompleteness theorems were the first of several 
closely related theorems on the limitations of formal systems. They were followed 
by Tarski's undefinability theoremon the formal undefinability of truth, Church's proof that 
Hilbert's Entscheidungsproblemis unsolvable, and Turing's theorem that there is no 
algorithm to solve the halting problem.
Formal systems : completeness , consistency and effective axiomatization 
This incompleteness is applicable to formal systems that are of sufficient complexity to 
express the basic arithmetic of natural numbers and which are consistent and effective 
axiomatization .
In general, a formal system is a deductive apparatus that consists of a particular set of 
axioms along with rules of symbolic manipulation (or rules of inference) that allow for 
the derivation of new theorems from the axioms.
Effective axiomatization :
It is recursively enumerable set .which means that there is a computer program that, in 
principle, could enumerate all the theorems of the system without listing any statements 
that are not theorems. Examples of effectively generated theories include Peano arithmetic 
and Zermelo–Fraenkel set theory(ZFC).
Consistency : 
A set of axioms is consistentif there is no statement such that both the statement and 
its negation are provable from the axioms, and inconsistentotherwise.

\section{Relationship with computability}
The incompleteness theorem is closely related to several results about undecidable 
sets in recursion theory.
Kleene presented a proof of Gödel's incompleteness theorem using basic results of 
computability theory. One such result shows that the halting problemis undecidable: 
there is no computer program that can correctly determine, given any program P as 
input, whether P eventually halts when run with a particular given input. Kleene 
showed that the existence of a complete effective system of a rithmetic with certain 
consistency properties would force the halting problem to be decidable, a contradiction
.
The incompleteness results affect the philosophy of mathematics, particularly version 
of formalism,which use a single system of formal logic to define their principles. The
incompleteness theorem is sometimes thought to have severe sequences for the 
program of localism
\end{document}
