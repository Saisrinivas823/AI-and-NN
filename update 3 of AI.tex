\documentclass{article}
\usepackage[utf8]{inputenc}

\title{update 3 of term paper}
\author{Submitted By Lingamgunta saikumar}
\date{18th oct 2021}

\begin{document}

\maketitle

\section{}
The conventional Electromyography (EMG) technique uses bipolar surface electrodes, placed over 
the muscle belly of the targeted group of muscles. The electrodes are noninvasive, inexpensive, and 
readily incorporated into the socket of the prosthesis. These surface electrode have limitations like 
inability to record the signal from different muscle group at a time, inconsistency in signal magnitude 
and frequency, due to change in skin electrode interface associated in physiological and 
environmental modifications and also the EMG signals may encounter noise and interference from 
other tissues. Apart from these limitations it is easy to use by amputee and risk free. The amplitude 
of the EMG signal is mostly proportional to the contraction of the remaining muscle. To enhance the 
quality of the signal the Myoelectric control of prosthesis or other system utilizes the electrical 
action potential of the residual limb’s muscles that are emitted during muscular contractions. These 
emissions are measurable on the skin surface at a microvolt level. The emissions are picked up by 
one or two electrodes and processed by band-pass filtering, rectifying, and low-pass filtering to get 
the envelope amplitude of EMG signal for use as control signals to the functional elements of the 
prosthesis. The myoelectric emissions are used only for control. In simultaneous control (muscle co 
contraction) and proportional control (fast and slow muscle contraction) controls the two different 
mode from wrist to terminal device and vice versa.
The advance method over the conventional technique of EMG signal which replace the complicated 
mode of switching is the pattern recognition. This new control approach is stranded on the 
assumption that an EMG pattern contains information about the proposed movements involved in a 
residual limb. Using a technique of pattern classification, a variety of different intended movements 
can be identified by distinguishing characteristics of EMG patterns. Once a pattern has been 
classified, the movement is implemented through the command sent to a prosthesis controller. EMG 
pattern-recognition-based prosthetic control method involves performing EMG measurement (to 
capture reliable and consistent myoelectric signals), feature extraction (to recollect the most 
important discriminating information from the EMG), classification (to predict one of a subset of 
intentional movements), and multifunctional prosthesis control (to implement the operation of 
prosthesis by the predicted class of movement) .
In pattern recognition control for a multifunctional prosthesis, multi-channel myoelectric recordings 
are needed to capture enough myoelectric pattern information. The number and placement of 
electrodes would mainly depend on how many classes of movements are demanded in a multifunctional prosthesis and how many residual muscles of an amputee are applicable for myoelectric 
control. For myoelectric transradial prostheses, the EMG signals are measured from residual muscles 
with a number of bipolar electrodes (8-16) which are placed on the circumference of the remaining 
forearm in which 8 of the 12 electrodes were uniformly placed around the proximal portion of the 
forearm and the other 4 electrodes were positioned on the distal end. A large circular electrode was 
placed on the elbow of the amputated arm as a ground.
For acquisition of EMG signal 50 Hz-60 Hz can be used to remove or reduce more low-frequency to 
increase the control stability of a multifunctional myoelectric prosthesis . EMG feature extraction is 
performed on windowed EMG data, all EMG recordings channels are segmented into a series of 
analysis windows either with or without time overlap (WL (window length) is 100-250 ms).
\end{document}
