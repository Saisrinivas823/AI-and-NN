\documentclass{article}
\usepackage[utf8]{inputenc}

\title{pseudoscience}
\author{Submitted By Lingamgunta saikumar}
\date{6th august 2021}

\begin{document}

\maketitle

\section{Introdution}
Pseudoscience is differentiated from science because-although it usually 
claims to be science-pseudoscience does not adhere to scientific standards,
such as the scientific method, falsifiability of claims, and Mertonian norms.
Alter definition: 
“From philosophers of silvio funtowicz and Jerome R.Ravetz”.
pseudo-science may be defined as one where the uncertainty of its inputs must be 
suppressed, lest they render its outputs totally indeterminate.
Pseudoscience consists of statements, beliefs, or practices that claim to be both 
scientific and factual but are incompatible with the scientific 
method. Pseudoscience is often characterized by contradictory, exaggerated 
or unfalsifiable claims; reliance on confirmation bias rather than rigorous attempts 
at refutation; lack of openness to evaluation by other experts; absence of 
systematic practices when developing hypotheses; and continued adherence long 
after the pseudoscientific hypotheses have been experimentally discredited.
The demarcation between science and pseudoscience has philosophical, political, 
and scientific implications. Differentiating science from pseudoscience has practical 
implications in the case of health care, expert testimony, environmental policies, 
and science education. Distinguishing scientific facts and theories from 
pseudoscientific beliefs, such as those found in climate change 
denial, astrology, alchemy, alternative medicine, occult beliefs, and creation 
science, is part of science education and literacy.
Pseudoscience can have dangerous effects. For example, pseudoscientific anti vaccine activism and promotion of homeopathic remedies as alternative disease 
treatments can result in people forgoing important medical treatments with 
demonstrable health benefits, leading to deaths and ill-health. Furthermore, people 
who refuse legitimate medical treatments to contagious diseases may put others at 
risk. Pseudoscientific theories about racial and ethnic classifications have led 
to racism and genocide.
The term pseudoscience is often considered pejorative particularly by purveyors of 
it, because it suggests something is being presented as science inaccurately or even 
deceptively. Those practicing or advocating pseudoscience therefore frequently 
dispute the characterization.
Criticism of pseudoscience:
Philosophers of science such as Paul Feyerabend argued that a distinction between 
science and nonscience is neither possible nor desirable. Among the issues which 
can make the distinction difficult is variable rates of evolution among the theories 
and methods of science in response to new data.

Larry Laudan has suggested pseudoscience has no scientific meaning and is mostly 
used to describe our emotions: "If we would stand up and be counted on the side 
of reason, we ought to drop terms like 'pseudo-science' and 'unscientific' from our 
vocabulary; they are just hollow phrases which do only emotive work for 
us". Likewise, Richard McNally states, 
"The term 'pseudoscience' has become little more than an inflammatory buzzword 
for quickly dismissing one's opponents in media sound-bites" and "When 
therapeutic entrepreneurs make claims on behalf of their interventions, we should 
not waste our time trying to determine whether their interventions qualify as 
pseudoscientific. Rather, we should ask them: How do you know that your 
intervention works? What is your evidence?
 

\end{document}
